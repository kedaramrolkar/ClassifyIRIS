\documentclass[11pt]{article}
\usepackage[margin=1in]{geometry}
\usepackage{amsmath}

\title{Gender Classification from Iris using Machine Learning Techniques}
\author{Kedar Amrolkar and Arpita Tugave}
\date{\today}

\begin{document}
\maketitle
\pagenumbering{arabic}

\section{Introduction}
\paragraph{}
Iris is one of the most discriminating feature possessed by Human Beings. It is currently one of the most accurate, reliable and widely used Biometric Identification Traits, enriched with distinctive Textural Features. Feature based matching surpasses Non-Feature based matching in computation and accuracy. Gender Classification is important in two aspects. Firstly, determining the gender of an imposter is helpful in Crime Investigations. Secondly, classifying the probe iris image reduces computation and boosts performance. Most of the current gender classification systems work by analyzing face images. But, very few research papers and journals have explored the usability of Iris.  To mention: Tapia et al \cite{Tapia} obtained an accuracy of 91.33\% using Uniform Local Binary Pattern with SVM Gaussian Kernel function classifier. Bansal et al \cite{Bansal} used SVM Classifier with Gaussian Kernel function to obtain 85\% accuracy. Thomas et al \cite{Thomas} obtained an accuracy of 75\% with C4.5 Decision Tree Classifier using Bagging Ensemble. Lagree et al \cite{Lagree} used SMO Support Vector Machine classifier to obtain 62\% accuracy. Keeping these previous works as reference we intend to come up with a better optimum Gender Classification algorithm. Our steps would be as follows: After successfully implementing Iris Segmentation, we will extract a rich textural feature vector and try it with different machine learning techniques to achieve maximum accuracy. After a successful implementation, we will try to use the same for Ethnicity Classification. 

\bibliographystyle{plain}
\bibliography{BiBliographyFile}{}

%
%\begin{thebibliography}{9}
%\bibitem{emergingthreat} 		%introduction of concept of different fields from different sites
%Danesh Irani, Steve Webb, Kang Li, Calton Pu: 
%\textit{Large Online Social Footprints -An Emerging Threat}. 
% 
%\bibitem{leakage}		% a good math representation of bits leakage from different places
%Danesh Irani, Steve Webb, Calton Pu, Kang Li: 
%\textit{Modeling Unintended Personal-Information Leakage from Multiple Online Social Networks}
% 
%\bibitem{privacypaper} %study leakage from different sites from the view point of third party apps accessing data
%Balachander Krishnamurthy, Craig E. Wills: 
%\textit{Characterizing Privacy in Online Social Networks}
%
%\bibitem{inforevelation}	%early paper for CMU students amount of info revelation
%Ralph Gross, Alessandro Acquisti: 
%\textit{Information Revelation and Privacy in Online Social Networks}
%
%\bibitem{statswebsite} 
%Statistics Portal at statista.com
%\\\texttt{http://www.statista.com/topics/1164/social-networks/}
%
%\bibitem{newsarticle} 
%Socialnomics
%\\\texttt{http://www.socialnomics.net/2014/03/04/the-shocking-truth-about-social-networking-crime/}
%
%\end{thebibliography}

\end{document}
